%%%%% DO NOT EDIT, MIGRATED TO GITHUB 
\documentclass{Notes}
\begin{document}
\vspace*{\fill}
\begin{center}
{\centering\huge{LitePog}}\\
\vspace*{2em}
{\centering\huge{MECH 423 Project Report}}\\
\vspace*{2em}
{\large Jasper Chan: 37467164 }\\
{\large Jacky Chang: 19835164 }    
\end{center}
\vspace*{\fill}
\newpage

\begin{abstract}
The objective of this project is to design and create a simple two player game similar to Light Pong\footnotemark. 
\footnotetext{
    \url{https://www.kickstarter.com/projects/aaqib/light-pong-game}
}
The system will consist of two handles/controllers on each end of a flexible tube that houses an LED strip.
The MSP430 will be used to control the LED's and process the signals from each controller.The mechanical housing was designed in Solidworks and printed in PLA using Fused Deposition Modelling (FDM).
The PCB's were custom designed and ordered from JLPCB.

The game was slightly different from the original LightPong.
Our version starts in an idle state and requires both players to hold down their buttons to start a countdown.
At the end of the countdown the first player to let go launches the "ball" and the light begins to move towards the opposing player. 
Players wait until the button on their handle lights up at which point they would need to press the button before a specified elapses.
If they fail to do so the game ends and they lose, if they do press the button in time, the light moves towards the other player at a slightly increased speed.
This process repeats until one player misses.

\end{abstract}

\section{Objectives}

The project is largely a re-implementation of the Light Pong Kickstarter.
All critical functions of the device were completed, including functional PCBs, a 3D printed housing, and the firmware implementing the game logic.

The only planned feature that was not completed on time was sourcing a translucent/transparent silicone tube to protect the LED strip and wiring.

\section{Rationale}

It's interesting to reverse engineer a product and see what type of difficulties the original developers may have experienced along the way.
The product is still currently only available for preorder, and units are not slated to ship until Q3 2022.
Because it can't be purchased, this is a good opportunity to have a similar device without waiting for preorders to start shipping.

\section{Summary of Functional Requirements}

\begin{table}[H]
    \centering
    \begin{tabularx}{\textwidth}{>{\hsize=.6\hsize\raggedright}X>{\hsize=.1\hsize\raggedright}X>{\hsize=.3\hsize\raggedright\arraybackslash}X}
        Task & Effort & Person Responsible  \\\toprule
         Power + charging PCB capable of delivering enough current for LED strip&20\% & Jasper\\\midrule
         Microcontroller PCB to drive LED strip and read button inputs & 20\% & Jasper\\\midrule
         Mechanical housing for parts& 20\% & Jacky\\\midrule
         Addressable LED Strip& 5\% & Jacky\\ \midrule
         Microcontroller code & 35\% & Jasper and Jacky\\\bottomrule
    \end{tabularx}
    \caption{Functional Requirements and their efforts}
    \label{tab:funcreq}
\end{table}

\section{Functional Requirement 1: Addressable LED strip}

\subsection{Approach and Design}

The most common addressable LED strips on the market are all based on the WS2812B (commonly referred to as Neopixels).
The general rule of thumb is that each LED may consume up to \SI{60}{\milli\ampere} of current, although this rarely happens when displaying patterns on the strip since not all LEDs will be on/at full power.
For our application, Adafruit sells\footnotemark\ a \SI{1}{\meter} long strip with 30 LEDs.
\footnotetext{\url{https://www.adafruit.com/product/1376}}
This provides an acceptable balance between LED density and current consumption.

These LEDs are typically rated for \SI{5}{\volt} power and logic, however they can commonly be driven at lower voltages.
In order to determine the power requirements for the power PCB, experimentation will have to be done.

\subsection{Inputs and Outputs}

\subsubsection{Inputs}

\begin{enumerate}
    \item Power 
    \begin{enumerate}
        \item Voltage: nominally \SI{5}{\volt}, actual TBD
        \item Current: Max \SI{60}{\milli\ampere} per LED 
    \end{enumerate}
    \item Data: the WS2812B data format is described in the datasheet\footnotemark
    \footnotetext{\url{https://cdn-shop.adafruit.com/datasheets/WS2812.pdf}}
    \item Wiring: Wires between the handles will need to be run along the LED strip
\end{enumerate}

\subsubsection{Outputs}

\begin{enumerate}
    \item Light: the LED strip will emit light
    \item Power: the power wiring inside the LED strip may be used to power the microcontroller PCB
    \item Wiring: Wires between the handles will need to be run along the LED strip
\end{enumerate}

\subsection{Parameters}

\begin{enumerate}
    \item Strip length
    \begin{enumerate}
        \item Adafruit sells strips in \SI{1}{\meter} segments, but they can be cut in between each LED
        \item A \SI{1}{\meter} strip will be used for convenience
    \end{enumerate}
    \item LED count
    \begin{enumerate}
        \item Adafruit sells strips at various LED densities
        \item Higher densities will increase power and memory requirements on the MCU
    \end{enumerate}
\end{enumerate}

\subsection{Development Plan}

For simplicity, a \SI{1}{\meter} strip with 30 LEDs will be used.

A \SI{3.3}{\volt} logic level microcontroller will be used to determine if the strip can be controlled without the use of a level shifter.

A bench power supply will be used to determine if a \SI{5}{\volt} boost circuit will be needed to power the LEDs.

\subsection{Test Plan}
\label{sec:fr1testplan}

\subsubsection{Logic level requirements}
\begin{enumerate}
    \item Power the LED strip with \SI{5}{\volt} from a bench power supply
    \item Connect a \SI{3.3}{\volt} logic level MCU to the data pin of the strip
    \item Attempt to display a test pattern, if this works, a level shifter is not required, if this does not work, redo the test with a level shifter.
\end{enumerate}

\subsubsection{Brightness requirements}
\begin{enumerate}
    \item Power the LED strip with \SI{5}{\volt} from a bench power supply
    \item Connect a MCU to the data pin of the strip
    \item Set all LEDs to full brightness white (RGB 255, 255, 255)
    \item Incrementally turn down the brightness until it is at a reasonable level
\end{enumerate}

\subsubsection{Voltage requirements}
\begin{enumerate}
    \item Power the LED strip with \SI{3.2}{\volt} from a bench power supply
    \item Connect a MCU to the data pin of the strip
    \item Set all LEDs to a reasonable brightness of white
    \item If the LEDs turn on correctly, note the current consumption, if not, repeat the test at a higher voltage until the LED strip reliably turns on.
    \begin{itemize}
        \item This will determine if we need a boost circuit or if the LEDs can be directly powered from a LiPo battery.
    \end{itemize}
\end{enumerate}
\subsubsection{Results}

The LED strip was capable of being controlled with \SI{3.3}{\volt} level logic signals from the MSP430.

It was determined that running the LEDs at about 100/255 brightness was a reasonable number that was clearly visible while keeping the current consumption managable.

Powering the LED strip at less than \SI{5}{\volt} was not tested since it was determined that a boost converter was necessary to utilize the entirety of the battery capacity.

\subsection{Integration}
\begin{enumerate}
    \item Power the LED strip with the determined acceptable voltage level
    \item Connect MCU (with level shifter if required) to the data pin of the strip
    \item Display a test pattern on the strip
\end{enumerate}

\section{Functional Requirement 2: Power + Charging PCB}
\label{sec:fr2}

\subsection{Approach and Design}

Because there is limited space to store a power source within the handles of the device, a custom PCB will have to be made to hold the battery, and associated charging, protection and boost circuitry as required.

Because the handle will be cylindrical, an 18650 battery will be used because it will be the most convenient form factor while providing high power and energy density.
For safety reasons, a spring loaded battery holder will be used to avoid soldering or spot welding connections.

\subsection{Inputs and Outputs}

\subsubsection{Inputs}

\begin{enumerate}
    \item Charging power: Battery will be charged with a USB connection
    \item Button: Power PCB will be housed in one of the device handles, and will need to have a connector for the handle button
    \item Button LED signal: Power PCB will need a connection to receive the signal to turn the button's LED on or off.
    \item Power Switch: The power PCB will need a power switch to disable the power output.
\end{enumerate}

\subsubsection{Outputs}

\begin{enumerate}
    \item LED strip power: Power PCB will need to provide power to the LED strip
    \item Button signal: Power PCB will need to have a connection to deliver the button signal to the MCU
    \item Button LED power: Power PCB will need to light up the button's LED when indicated by the MCU
\end{enumerate}

\subsection{Parameters}

\begin{enumerate}
    \item LED strip voltage: Determined in \autoref{sec:fr1testplan}
    \item Total sustained current: Determined in \autoref{sec:fr1testplan}
    \item Battery choice: Must be capable of providing enough power
    \begin{itemize}
        \item Battery may or may not come with protection circuitry built in.
        Requirement for protection circuitry on the PCB is only required for unprotected cells.
        PCB silkscreen will clearly indicate whether protected or unprotected cells should be used.
    \end{itemize}
    \item Charging current: for safety reasons this will be a conservative number (max 1C)
    \item PCB dimensions: PCB must be big enough to hold all the components
\end{enumerate}

\subsection{Development Plan}

\begin{enumerate}
    \item Design schematic and layout of PCB
    \item Send design to PCB house for manufacturing, buy components
    \item Assemble PCB
\end{enumerate}

\subsection{Test Plan}

\subsubsection{Charging}

\begin{enumerate}
    \item Place PCB and USB battery bank inside of a fireproof container
    \item Connect USB battery bank to PCB charging input
    \item Wait for charging to complete.
    \item Remove LiPo cell from PCB, ensure that it has the correct voltage for a charged cell (approximately \SI{4.2}{\volt})
\end{enumerate}

\subsubsection{Power Output} 

\begin{enumerate}
    \item Place PCB inside of a fireproof container
    \item Connect PCB power output to a load
        \begin{itemize}
            \item If the lab contains a programmable load, this is preferred, otherwise the load can just be the LED strip + MCU
        \end{itemize}
    \item Enable PCB power output and load
    \item Ensure power output is stable
    \item Keep test running, ensure power output stops at or before designed battery cutoff voltage is reached.
\end{enumerate}

\subsubsection{Button LED}

\begin{enumerate}
    \item Connect button to PCB
    \item Connect MCU GPIO to PCB
    \item Enable PCB power output
    \item Toggle Button LED GPIO, ensure LED toggles.
\end{enumerate}

\subsubsection{Button Input}

\begin{enumerate}
    \item Connect button to PCB
    \item Connect MCU GPIO to PCB
    \item Enable PCB power output
    \item Press and release button, ensure MCU can read button input.
\end{enumerate}

\subsubsection{Results}

After some modifications\footnotemark, the power PCB functions as expected.
\footnotetext{Modifications are outlined in project files.}

The power PCB was capable of providing \SI{5}{\volt} at \SI{1}{\ampere} continuously for over \SI{10}{\second}, which is more than we ever expect in normal use.
Battery over and undervoltage protections worked as expected.
One modification involves running the battery current through the power switch, which exceeds its \SI{500}{\milli\ampere} rating, however no heat or adverse effects were noticed even during prolonged use.

\section{Functional Requirement 3: Microcontroller PCB}

\subsection{Approach and Design}

As required by the class, an MSP430 will be used.
For simplicity, the same MSP430FR5739IDAR\footnotemark as used in Lab 3 will be used, and the design will be derived from the Lab 3 design.
\footnotetext{
    \url{https://www.digikey.ca/en/products/detail/texas-instruments/MSP430FR5739IDAR/2696032}
}

\subsection{Inputs and Outputs}

\subsubsection{Inputs}

\begin{enumerate}
    \item Power: Power will be provided from the PCB board, either via the LED strip, or via a separate wire. Power may have to be stepped down to safely power the MCU
    \item Button x2: MCU PCB will be housed in one of the device handles, and will need to have a connector for the handle button, as well as for the button on the other handle
\end{enumerate}

\subsubsection{Outputs}

\begin{enumerate}
    \item LED strip data: MCU PCB will need to control the LED strip
    \item Button LED power x2: MCU PCB will need to light up the button's LED when indicated by the MCU
\end{enumerate}

\subsection{Parameters}

\begin{enumerate}
    \item Crystal choice: Determine frequency and precision of MCLK source
\end{enumerate}

\subsection{Development Plan}

\begin{enumerate}
    \item Design schematic and layout of PCB
    \item Send design to PCB house for manufacturing, buy components
    \item Assemble PCB
\end{enumerate}

\subsection{Test Plan}

\begin{enumerate}
    \item Power up PCB with bench power supply (use low current limit to minimize risk of shorts)
    \item Check that MCU can be programmed
    \item Program MCU with test firmware
    \item Confirm functionality of all required IO pins with a test program
\end{enumerate}

\subsubsection{Results}

After some modifications\footnotemark, the microcontroller PCB functions as expected.
\footnotetext{Modifications are outlined in project files.}

\section{Functional Requirement 4: Mechanical Housing}

\subsection{Approach and Design}

A custom mechanical housing will have to be designed to house the other components.
To avoid spending too much time or money machining, 3D printing will be primarily used to fabricate mechanical parts.
To make debugging of other components easier, the housing will be designed to be disassembled easily.

\subsection{Inputs and Outputs}

\subsubsection{Inputs}

\begin{enumerate}
    \item Other components: all other components will either be fixed to or contained within the mechanical housing
\end{enumerate}

\subsubsection{Outputs}

\begin{enumerate}
    \item Mechanical stability: parts must be held securely
\end{enumerate}

\subsection{Parameters}

\begin{enumerate}
    \item Handle diameter: diameter must be small enough to be comfortable, but large enough contain the PCBs
    \item Handle length: length must be short enough to be comfortable, but long enough to contain the PCBs
\end{enumerate}

\subsection{Development Plan}

\begin{enumerate}
    \item Once PCBs have been designed, PCB mechanical models will be passed on to the mechanical housing designer
    \item Mechanical housing will be designed around the PCB models
\end{enumerate}

\subsection{Test Plan}

\begin{enumerate}
    \item PCBs (or printed models of the PCBs) will be assembled into the mechanical housings to test fit
\end{enumerate}

\subsubsection{Results}

PCBs were assembled successfully into the housings.

\section{Functional Requirement 5: Microcontroller Code}

\subsection{Approach and Design}

The code for the MCU will need to be written to implement the game behavior.
Once the schematic for the MCU PCB is finalized, code can be written to run on it.

\subsection{Inputs and Outputs}

\subsubsection{Inputs}

\begin{enumerate}
    \item Buttons: The code must read the button inputs at the appropriate time
\end{enumerate}

\subsubsection{Outputs}

\begin{enumerate}
    \item Button LEDs: The code must drive the button LEDs at the appropriate time
    \item LED strip: The code must drive the Neopixel LED strip at the appropriate time with the correct pattern
\end{enumerate}

\subsection{Parameters}

\begin{enumerate}
    \item Game speed: The movement speed of the "ball" will control how hard the game is to play
\end{enumerate}

\subsection{Development Plan}

\begin{enumerate}
    \item Once the schematic is finalized, code can be written for the PCB
\end{enumerate}

\subsection{Test Plan}

\begin{enumerate}
    \item Flash the code on the MCU PCB (or another MSP430 board if the MCU PCB has not been completed)
    \item Hook up the buttons and LED strip and test that the code behaves as expected
\end{enumerate}

\subsubsection{Results}

The code worked as expected, although play testing indicates that changes could be made to improve user experience.

\section{System Evaluation}

See the Results subsections within each Functional Requirements section.

\section{Reflections}

The project for the most part went very smoothly.
We played to our strong points which allowed us to progress quickly; however, it may have proved to be a better learning experience if we worked on our weaker areas. 
Some things that we would have done differently is speccing and ordering things earlier.
As a result of this oversight, we were unable to obtain a tube to house the LED strip.
Although the performance didn't suffer, it was still something that we would have liked to have.

The most important things we learned in MECH 423 that we deemed to be the most important are:
\begin{itemize}
    \item Learning how to properly read and interpret documentation 
    \item Writing and debugging code
    \item PCB design
\end{itemize}
We picked these three aspects of the course because they are generic topics that are more applicable to the working world as opposed to how to use the MSP430 specifically.
Furthermore, these are things that we believe will be very common for us to use and refer back to as mechatronic engineers; especially within the field of research and development.

Currently, we lack experience implementing mechanical systems.
We have not had very much experience either designing or building a physical system within courses or co-op.
Moving forward, it would be nice to get more practice going through the entire design cycle for a mechanical system so that we may be more proficient in things like designing test jigs or final products.
To acquire this experience, we plan to pursue personal projects with enough difficulty that would challenge our current knowledge and allow us to grow.
We can also achieve this by working in a field that requires rapid iteration and design like a start-up or a research and development position.
\end{document}
